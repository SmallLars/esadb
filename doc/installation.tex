\chapter{Installation}

\section{Voraussetzungen}

\subsection{Linien}
\begin{itemize}
  \item ESA 2002 (1.3.1024)
  \item Microsoft Jet 3.5 für den Datenbankzugriff \newline http://download.microsoft.com/download/office97pro/sp/1/win98/EN-US/Jet35sp3.exe
  \item Bei der Nutzung von Windows 7 ist es empfehlenswert einige Registry-Einträge zu setzen, um die Steuerung mit esadb zu beschleunigen. \newline \newline
        HKEY\_LOCAL\_MACHINE\textbackslash SYSTEM\textbackslash CurrentControlSet\textbackslash services\textbackslash LanmanWorkstation\textbackslash Parameters
        \begin{itemize}
         \item DirectoryCacheLifetime = 0
         \item FileNotFoundCacheLifetime = 0
         \item FileInfoCacheLifetime = 0
        \end{itemize}
\end{itemize}
        
\subsection{Server}
\begin{itemize}
  \item Java 8 \newline http://java.com/de/download/
  \item Microsoft Office Access 2007-Runtime \newline https://www.microsoft.com/de-de/download/details.aspx?id=4438
  \item Office Microsoft Office Access Runtime und Data Connectivity 2007 Service Pack 3 (SP3) \newline https://www.microsoft.com/de-de/download/details.aspx?id=27835
\end{itemize}

\section{Anleitung}
Im Allgemeinen ist HAWEV unter "`C:\textbackslash esa\_dat\textbackslash HAWEV"' bereits installiert.
"`C:\textbackslash esa\_dat"' ist dabei der freigegebene Ordner der auf den Linien unter
"`Z:"' als \keyword{Netzlaufwerk} eingebunden ist.

Um keine Änderung an den Linien vornehmen zu müssen, kann der Ordner "`HAWEV"' umbenannt werden, um für esadb einen
neuen Ordner "`HAWEV"' zu erstellen, in den dann die esadb.jar kopiert wird. Danach sind folgende Schritte empfohlen:
\begin{enumerate}
  \item esadb.jar per Doppelklick starten
  \item Unter "`Datei $ \Rightarrow $ Einstellungen... $ \Rightarrow $ Allgemein"' die gewünschten Linien einfügen
  \item esadb beenden
  \item Für die Nutzung der Wildscheiben müssen die Dateien HZ\_7775 und *.bmp auf die Linien in das Installationsverzeichnis von ESA 2002 kopiert werden.
  \item Eine Verknüpfung zu esadb.jar auf dem Desktop anlegen
  \item Eine Verknüpfung zu Stammdaten.mdb auf dem Desktop anlegen
  \item Vereine, Schützen und Disziplinen in den Stammdaten hinterlegen
\end{enumerate}

\section{Programmdateien}
\begin{description}
  \item[HZ\_7775 und *.bmp] sind Bilddateien für \keyword{Wildscheibe}n (\keyword{Trefferzonenscheibe}n).
                           Sie werden für die Anzeige und Auswertung der Schüsse benötigt und müssen auch
                           in den Installationsverzeichnissen von ESA 2002 auf den Linien vorhanden sein.
  \item[*.esa] ist eine Wettkampfdatei. Der Inhalt wird näher erläutert in Abschnitt \ref{sec:filehandling}.
  \item[settings.esc] beinhaltet alle Einstellungen. Die Einstellungen gelten für alle Wettkämpfe.
                      Eine mögliche Datensicherung sollte diese Datei enthalten.
  \item[esadb.ico] kann als Symbol für eine Verknüpfung auf dem Desktop oder im Startmenü genutzt werden.
  \item[logfile.txt] enthält ein Protokoll aller Programmausgaben und Fehlermeldungen.
  \item[Stammdaten.mdb] ist ein Programm zur Verwaltung der Stammdaten. Diese Datei enthält selber keine
                        Daten und wird regelmäßig aktualisiert.
  \item[data.mdb] enthält alle Stammdaten, die mit Hilfe der Stammdaten.mdb eingegeben wurden. Eine mögliche
                  Datensicherung sollte diese Datei enthalten.
  \item[Kampf.mdb] ist eine Kopie von data.mdb und wird von den Linien ausgelesen um Disziplin- und Schützendaten abzurufen.
  \item[esadb.lock] ist nur vorhanden, solange esadb läuft. Sie dient dazu einen mehrfachen Start von esadb zu verhindern.
  \item[*.ctl und *.nrt] werden genutzt um den Linien Steuerbefehle zu senden.
  \item[*.def] enthalten Regel-, Scheiben- und Waffen-Definitionen. Sie werden bei Bedarf automatisch erstellt und
               bei beenden von esadb bereinigt.
\end{description}
